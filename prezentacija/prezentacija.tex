\documentclass{beamer}
%
% Choose how your presentation looks.
%
% For more themes, color themes and font themes, see:
% http://deic.uab.es/~iblanes/beamer_gallery/index_by_theme.html
%
\mode<presentation>
{
  \usetheme{Frankfurt}      % or try Darmstadt, Madrid, Warsaw, ...
  \usecolortheme{seahorse} % or try albatross, beaver, crane, ...
  \usefonttheme{default}  % or try serif, structurebold, ...
  \setbeamertemplate{navigation symbols}{}
  \setbeamertemplate{caption}[numbered]
} 

\usepackage[utf8]{inputenc}
\usepackage[T2A]{fontenc} % enable Cyrillic fonts
\usepackage[english,serbianc]{babel} %ukljuciti babel sa ovim opcijama, umesto gornjim, ukoliko se koristi cirilica

\title[Генетичко програмирање]{Генетичко програмирање}
\author{Александра Стојановић, Ивана Ивановић,\\ Александар Стефановић, Оливера Поповић}
\institute{Математички факултет}
\date{Date of Presentation}

\begin{document}

\begin{frame}
  \titlepage
\end{frame}

% Uncomment these lines for an automatically generated outline.
\begin{frame}{Садржај}
  \tableofcontents
\end{frame}

\section{Увод}

	\begin{frame}{Увод}

	\begin{itemize}
	  \item Аутоматизација прављења рачунарских програма
	  \item Познато је једино шта треба да буде урађено
	\end{itemize}

	\end{frame}

\section{Историјат}

	\begin{frame}{Историјат}
		\begin{itemize}
			\item 1948 — Алан Тјуринг
     		\item 1962 — Џон Холанд	
     		\item 1985 — Крамер
     		\item 1989 — Коза
		\end{itemize}
	\end{frame}

	\begin{frame}{Историјат}
		\begin{itemize}
			\item Низвои битова фиксне дужине да представе бројеве у проблемима оптимизације
     		\item Репрезентација променљиве дужине
     		\item Репрезентација помоћу стабала
		\end{itemize}
	\end{frame}

\section{Опис алгоритма}

	\begin{frame}{Опис алгоритма}
		\begin{itemize}
			\item Инспирисан процесом еволуције у природи
     		\item Спада у групу алгоритама \textbf{еволутивног израчунавања}	
     		\item Процес природне селекције
     		\item Комбиновање г
		\end{itemize}
	\end{frame}

	\begin{frame}{Опис алгоритма}
	\end{frame}

	\begin{frame}{Опис алгоритма}
	\end{frame}

	\begin{frame}{Опис алгоритма}
	\end{frame}
	
\section{Примери примене}

	\begin{frame}{Примери примене}
	\end{frame}

	\begin{frame}{Примери примене}
	\end{frame}
	
\section{Мета-генетичко програмирање}

	\begin{frame}{Мета-генетичко програмирање}
	\end{frame}

	\begin{frame}{Мета-генетичко програмирање}
	\end{frame}
	
\section{Литература}

	\begin{frame}{Литература}
	\end{frame}





































\iffalse
\section{Some \LaTeX{} Examples}

\subsection{Tables and Figures}

\begin{frame}{Tables and Figures}

\begin{itemize}
\item Use \texttt{tabular} for basic tables --- see Table~\ref{tab:widgets}, for example.
\item You can upload a figure (JPEG, PNG or PDF) using the files menu. 
\item To include it in your document, use the \texttt{includegraphics} command (see the comment below in the source code).
\end{itemize}

% Commands to include a figure:
%\begin{figure}
%\includegraphics[width=\textwidth]{your-figure's-file-name}
%\caption{\label{fig:your-figure}Caption goes here.}
%\end{figure}

\begin{table}
\centering
\begin{tabular}{l|r}
Item & Quantity \\\hline
Widgets & 42 \\
Gadgets & 13
\end{tabular}
\caption{\label{tab:widgets}An example table.}
\end{table}

\end{frame}

\subsection{Mathematics}

\begin{frame}{Readable Mathematics}

Let $X_1, X_2, \ldots, X_n$ be a sequence of independent and identically distributed random variables with $\text{E}[X_i] = \mu$ and $\text{Var}[X_i] = \sigma^2 < \infty$, and let
\[ S_n = \frac{X_1 + X_2 + \cdots + X_n}{n}
      = \frac{1}{n}\sum_{i}^{n} X_i \]
denote their mean. Then as $n$ approaches infinity, the random variables $\sqrt{n}(S_n - \mu)$ converge in distribution to a normal $\mathcal{N}(0, \sigma^2)$.

\end{frame}
\fi

\end{document}
